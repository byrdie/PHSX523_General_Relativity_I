% --------------------------------------------------------------
% This is all preamble stuff that you don't have to worry about.
% Head down to where it says "Start here"
% --------------------------------------------------------------
 
\documentclass[10pt]{article}
 
\usepackage[margin=.3in, voffset=.3in, ]{geometry} 
\usepackage{amsmath,amsthm,amssymb, mathtools}
\usepackage{multicol}
\usepackage[subnum]{cases}
\usepackage{relsize}
\usepackage[makeroom]{cancel}
\usepackage[english]{babel}
\usepackage{graphicx}
\usepackage{calligra}
\usepackage[normalem]{ulem}
\usepackage{caption}
\usepackage{subcaption}
\usepackage{fancyhdr}
\usepackage{mathrsfs}
\usepackage{bbold}


\DeclareMathAlphabet{\mathcalligra}{T1}{calligra}{m}{n} 
\DeclareFontShape{T1}{calligra}{m}{n}{<->s*[2.2]callig15}{}


% Makes '\sr' make a script r
\newcommand{\sr}{\ensuremath{\mathcalligra{r}}}
 
\newcommand{\N}{\mathbb{N}}
\newcommand{\Z}{\mathbb{Z}}
\newcommand{\ihat}{\boldsymbol{\hat{\textbf{\i}}}}
\newcommand{\jhat}{\boldsymbol{\hat{\textbf{\j}}}}
\newcommand{\khat}{\boldsymbol{\hat{\textbf{k}}}}
\newcommand{\rhat}{\boldsymbol{\hat{\textbf{r}}}}
\newcommand{\srhat}{\boldsymbol{\hat{\textbf{\sr}}}}
\newcommand{\xhat}{\boldsymbol{\hat{\textbf{x}}}}
\newcommand{\yhat}{\boldsymbol{\hat{\textbf{y}}}}
\newcommand{\zhat}{\boldsymbol{\hat{\textbf{z}}}}
\newcommand{\nhat}{\boldsymbol{\hat{\textbf{n}}}}
\newcommand{\phihat}{\boldsymbol{\hat{\textbf{$\phi$}}}}
\newcommand{\thetahat}{\boldsymbol{\hat{\textbf{$\theta$}}}}
\newcommand{\rhohat}{\boldsymbol{\hat{\textbf{$\rho$}}}}

\newcommand{\ve}[1]{\overrightarrow{#1}}
\newcommand{\vect}[1]{\boldsymbol{\mathbf{#1}}}
\newcommand{\vc}[1]{\overrightarrow{#1}}
\newcommand{\fracl}[2]{\mathlarger{\frac{#1}{#2}}}
\newcommand{\dd}{\, \mathrm{d}}
\newcommand{\eo}{\epsilon_0}
\newcommand{\mo}{\mu_\circ}
\newcommand{\tder}[2]{\frac{\dd #1}{\dd #2}}
\newcommand{\pder}[2]{\frac{\partial #1}{\partial #2}}
\newcommand{\dtder}[2]{\frac{\dd^2 #1}{\dd #2^2}}
\newcommand{\ttder}[2]{\frac{\dd^3 #1}{\dd #2^3}}
\newcommand{\dpder}[2]{\frac{\partial^2 #1}{\partial #2^2}}
\newcommand{\tpder}[2]{\frac{\partial^3 #1}{\partial #2^3}}
\newcommand{\intas}{ \int_{-\infty}^\infty}
\newcommand{\wt}[1]{\widetilde{#1}}
\newcommand{\ev}[1]{\left\langle #1 \right\rangle}
\newcommand{\ce}{\wt{\vect{E}}}
\newcommand{\cb}{\wt{\vect{B}}}
\newcommand{\K}{\frac{1}{4 \pi \eo}}
\newcommand{\lrp}[1]{\left( #1 \right)}
\newcommand{\lrb}[1]{\left[ #1 \right]}
\newcommand{\lrc}[1]{\left\{ #1 \right\}}
 
\newenvironment{theorem}[2][Theorem]{\begin{trivlist}
\item[\hskip \labelsep {\bfseries #1}\hskip \labelsep {\bfseries #2.}]}{\end{trivlist}}
\newenvironment{lemma}[2][Lemma]{\begin{trivlist}
\item[\hskip \labelsep {\bfseries #1}\hskip \labelsep {\bfseries #2.}]}{\end{trivlist}}
\newenvironment{exercise}[2][Exercise]{\begin{trivlist}
\item[\hskip \labelsep {\bfseries #1}\hskip \labelsep {\bfseries #2.}]}{\end{trivlist}}
\newenvironment{problem}[2][Problem]{\begin{trivlist}
\item[\hskip \labelsep {\bfseries #1}\hskip \labelsep {\bfseries #2.}]}{\end{trivlist}}
\newenvironment{question}[2][Question]{\begin{trivlist}
\item[\hskip \labelsep {\bfseries #1}\hskip \labelsep {\bfseries #2.}]}{\end{trivlist}}
\newenvironment{corollary}[2][Corollary]{\begin{trivlist}
\item[\hskip \labelsep {\bfseries #1}\hskip \labelsep {\bfseries #2.}]}{\end{trivlist}}


\newenvironment{Figure}
  {\par\medskip\noindent\minipage{\linewidth}}
  {\endminipage\par\medskip}

\pagenumbering{gobble}

\pagestyle{fancy}
\lhead{Midterm 1}
\chead{PHSX 523 General Relativity I}
\rhead{Roy Smart}


 
\begin{document}
\small
\begin{multicols}{2}
	\setlength{\abovedisplayskip}{-25pt}
	\setlength{\belowdisplayskip}{0pt}
	\setlength{\abovedisplayshortskip}{0pt}
	\setlength{\belowdisplayshortskip}{0pt}
	\begin{align*}
		& \small \hspace{-10pt} \textbf{Special Relativity} \small \\
			& \Delta s^2 = -\Delta t^2 + \Delta x^2 +\Delta y^2 + \Delta z^2	\tag*{Spacetime interval (S. 1.1)} \\
			& \Delta \overline{s}^2 = \Delta s^2	\tag*{Invariance of the interval (S. 1.7)} \\
			\begin{split}
				& \overline{t} = \gamma (t - v x) \\
				& \overline{x} = \gamma (v - vt) \\	
			\end{split} 	\tag*{Lorentz Transforms (S. 1.12)} \\
			& \Delta x^{\overline{\alpha}} = \Lambda_{\ \beta}^{\overline{\alpha}} \Delta x^\beta \tag*{Lorentz Transformation (S. 2.4)} \\
			& \vec{A} \xrightarrow[\mathcal{O}]{} A^\alpha	\tag*{Components of the vector $\ve{A}$ (S. 2.7)} \\
			& \lrp{\vec{e}_\alpha}^\beta = \delta_{\alpha}^{\ \beta}		\tag*{Definition of the basis vectors (S. 2.10)} \\
			& \vec{A} \rightarrow A^\alpha \vec{e}_\alpha	\tag*{The vector $\vec{A}$ in terms of the basis vectors (S. 2.10)} \\
			& \vec{e}_\alpha = \Lambda_{\ \alpha}^{\overline{\beta}} \vec{e}_{\overline{\beta}}		\tag*{Lorentz transform of basis vectors (S. 2.13)} \\
			& \Lambda_{\ \alpha}^{\overline{\beta}} = \begin{pmatrix}
				\gamma & -v \gamma & 0 & 0 \\
				-v \gamma & \gamma & 0 & 0 \\
				0 & 0 & 1 & 0 \\
				0 & 0 & 0 & 1 \\
			\end{pmatrix}	\tag*{Components of the $\Lambda$ tensor (S. p38)} \\
			& \vec{e}_{\ \overline{\mu}}^\nu(-\vec{v})e_\vec{e}_\nu	\tag*{Inverse Lorentz transform (S. 2.15)} \\
			& \vec{e}_\alpha = \delta_{\ \alpha}^\nu \vec{e}_\nu		\tag*{Basis vector identity (S p40)} \\
			& \Lambda_{\ \overline{\beta}}^\nu \\
	\end{align*}
	\setlength{\abovedisplayskip}{-25pt}
	\setlength{\belowdisplayskip}{0pt}
	\setlength{\abovedisplayshortskip}{0pt}
	\setlength{\belowdisplayshortskip}{0pt}
	\begin{align*} 	
	\end{align*}

\end{multicols}
% --------------------------------------------------------------
%     You don't have to mess with anything below this line.
% --------------------------------------------------------------
 
\end{document}
